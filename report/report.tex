\documentclass{../cs-classes/cs-classes}

\title{Reasoning and Provenance\\ on Neural Networks}
\author{
    \textbf{Antoine Groudiev}\\L3, DI, ENS
    \and
    \textbf{Silviu Maniu}\\Slide Team, LIG, UGA
}

\addbibresource{report.bib}
\graphicspath{{../cs-classes}}

\begin{document}
\begin{abstract}
    Recently, neural networks allowed computers to solve numerous problems from diverse machine learning fields, such as natural language processsing and computer vision. Compared to traditional algorithms, machine learning models have proven both more successful and more difficult to interpret. Neural networks are considered as black boxes unable to easily explain itself, that is justifying the reasons that led him to make a prediction. Layer-wise Relevance Propagation (LRP) is a technique that has been introduced to provide explanability by identifying the input features relevant to the output choice. In parallel, research in the databases field developed annotations techniques to compute provenance for queries. In this paper, we extend LRP propagation rules to semiring-based provenance annotations of the network, for LRP to gain in expressivity.
\end{abstract}

\section{Introduction}
\subsection{Problem statement}

\subsection{Layer-wise Relevance Propagation}

\subsection{Semiring-based provenance annotations}

\newpage
\nocite{*}
\printbibliography

\end{document}